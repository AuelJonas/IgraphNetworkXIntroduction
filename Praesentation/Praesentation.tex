\documentclass[12pt]{beamer}
\usetheme{Warsaw}
\setbeamertemplate{headline}{}
\usecolortheme{seahorse}
\usepackage[utf8]{inputenc}
\usepackage[T1]{fontenc}
\usepackage{lmodern}
\usepackage{graphicx}
\usepackage{wrapfig}
\usepackage{epstopdf} 
\usepackage{epsfig} 
\usepackage{psfrag} 
\usepackage{pgfplots} 
\usepackage{multirow}
\usepackage{rotating}

\author{Jonas Auel, Sven Fritz}
\title{NetworkX and igraph}
\subtitle{Graph Analysis using Python}
%\logo{}
\institute{Uni Mannheim}
\date{October 28th 2016}
\newtheorem*{satz}{Satz}
\newtheorem*{defi}{Definition}
%\subject{}
%\setbeamercovered{transparent}
%\setbeamertemplate{navigation symbols}{}

\begin{document}
	\maketitle
	%\frame{\tableofcontents\frametitle{Overview}}
	\section{Why Python?}
	\begin{frame}{Why Python?}
		\begin{itemize}
			\item powerful programming languages
			\item allows clear and concise expressions of network algorithms
			\item growing ecosystem of packages that provide more features
			%\item Python is an excellent "glue" language for putting together pieces of software
			\item provides packages in many fields, such as machine learning, statistics and numerics
			\item in the U.S. Python is by now the most popular programming language for introduction courses
		\end{itemize}
	\end{frame}
	\begin{frame}{NetworkX}
		\begin{itemize}
			\item Network creation, manipulation, analyzation (and visualization)
			\item available for Python
			\item supported platforms: Linux/Windows/Mac
			\item load and store networks in standard and nonstandard data formats
			\item nodes can be "anything" (e.g. images)
			\item edges can hold arbitrary data (e.g. time series)
			\item open source
		\end{itemize}
	\end{frame}
	\begin{frame}{igraph}
		\begin{itemize}
			\item Network creation, manipulation, analyzation and visualization
			\item available for C/R/Python
			\item supported platforms: Linux/Windows/Mac 
			\item collection of graph analysis tools
			\item emphasis on efficiency, portability, ease of use
			\item open source
		\end{itemize}
	\end{frame}
	\begin{frame}{Graph types in NetworkX and igraph}
		\centering
		\begin{tabular}[]{c|c|c}
			Graph type & NetworkX class & igraph class \\
			\hline
			Undirected & Graph & Graph \\
			Directed & DiGraph & Graph \\
			With self-loops & Graph, DiGraph & Graph \\
			With parallel edges & MultiGraph, MultiDiGraph & Graph
		\end{tabular}
	\end{frame}
	\begin{frame}{Betweenness centrality}
		\begin{itemize}
			\item Betweenness centrality of a node $v$: sum of the
			fraction of all-pairs shortest paths that pass through $v$
			\item 		
			\begin{equation*}
			c_B(v) =\sum_{s,t \in V} \frac{\sigma(s, t|v)}{\sigma(s, t)}
			\end{equation*}	
			\item  $V$: set of nodes,\\
			$\sigma(s, t)$: number of
			shortest $(s, t)$-paths, \\
			$\sigma(s, t|v)$: number of those
			paths  passing through some  node $v$ other than $s, t$
			\item if $s = t$, $\sigma(s, t) = 1$, and if $v \in {s, t}$,
			$\sigma(s, t|v) = 0$.
		\end{itemize}
	\end{frame}
	\begin{frame}{hub score}
		\begin{itemize}
			\item test
		\end{itemize}
	\end{frame}
\end{document}
